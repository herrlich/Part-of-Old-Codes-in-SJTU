%%==================================================
%% chapter03.tex for SJTU Bachelor Thesis
%% version: 0.5.2
%% Encoding: UTF-8
%%==================================================

% \bibliographystyle{sjtu2} %[此处用于每章都生产参考文献]

\chapter{Related Works}
\label{chap:related}

Since the amount of data needed in real networks are dramatically increasing, tons of researches have been done to increase the efficiency of current DCNs \cite{dcnarch,hedera,spain,vl2,elastictree,flyways}. And in order to make the management of DCN effective and easy, several structures are also raised to deploy SDN in data centers, such as Global Environment for Network Innovations (GENI) of US, Future Internet Research and Experimentation Initative (FIRE) of EU, AKARI \& JGN2plus of Japan, and Smart Applications on Virtual Infrastructure (SAVI) of Canada. Among these works, OpenFlow is the most popular one, which uses software controllers to manipulate the whole system.

In early data centers that deploys such structures, a centralized controller is usually used to control the whole system. A DCN that uses a centralized controller is very easy to deploy, and also saves the energy of many controllers. \cite{hedera,spain,openflow,elastictree} are examples of this kind. They can get the global information of the network, and then make appropriate choices when flows come in. It is obvious that a centralized controller is enough for a data center that is not very large, and it also simplifies the process of making decisions.

Nevertheless, a simple centralized controller can also easily becomes the bottleneck of the whole system. When the DCN is very large and complex, the performance of the system will be influenced by the limited capacity of this single controller. Thus, researchers raised the idea of devolved controllers \cite{devolved,devolvedglobecom,elasticsdn,hyperflow,inpacket,multictr} to solve the problem of scalability. Using devolved controllers, the system will be split into several parts, and each devolved controller will work as a centralized controller in its own region.

At the same time, some other scholars also did some research \cite{localcen,elasticsdn,hyperflow,inpacket,onix} on distributed controllers for this SDN structure, to solve the problem of scalability in centralized structure. These researches discuss the statically configured and dynamically configured mapping between a switch and a controller in SDN. Software Defined Networking (SDN) enables the programmability of the network, and brings easier control and fantanstic innovation \cite{openflow,ethane,soft}. In SDN standard that specified by Open Networking Foundation \cite{onf}, the control plan and the data plane are seperated. Currently, one widely used protocol in SDN is OpenFlow \cite{openflow}, which is raised by Stanford University. In OpenFlow, the control plane can manipulate the flows in the data plane by using programmable controllers. And as we mentioned above, a single centralized controller will become the bottleneck of the system, thus works such as HyperFlow \cite{hyperflow}, DevoFlow \cite{DevoFlow}, DAIM \cite{DAIM} and Devolved Controllers \cite{devolved} are attempting to address this issue, by either using multiple controllers or devolving network control to switches. And all these methods requires a global view of the network. Comparatively, a hierarchical structure Kandoo \cite{kandoo} only needs local information in its bottom layer, and requires global information in its top layer.

Except for these theoretical researches, may works are done in practical area. Authors in \cite{DOT} present a low cost network emulator called Distributed OpenFlow Testbed (DOT), which can emulate large SDN deployments, thus making it convenient for scholars to test their schemes. And Google also published their B4 \cite{B4} network, which is established on a DCN that is based on OpenFlow. B4 is devided into three layers: switch hardware layer, site controllers layer and global layer. Each site is a data center in B4. The switches and controllers are deployed in every data center, while the SDN gateway and the Traffic Engineering (TE) servers are deployed in a centralized area. Another migration protocol based on OpenFlow is presented by the authors in \cite{elasticsdn}. By all these achievements above, the commercialized ultilization of OpenFlow is currently on its right track.
